\section{Conclusion}
\label{sec:conclusion}

This paper examines how the fragmentation of land use authority within cities affects housing supply and prices. I study Chicago's ``aldermanic privilege'': the informal institution granting each of the city's 50 aldermen effective veto power over development in their ward—and document its consequences for housing markets.

I construct aldermanic strictness scores from permit processing times, using a residualization approach that removes the influence of ward characteristics and an empirical Bayes procedure that addresses measurement error for aldermen with few permits. The resulting scores reveal substantial and persistent variation in regulatory stringency across aldermen that is not simply a reflection of geography or neighborhood demographics.

Using these scores, I employ two complementary empirical strategies. First, a spatial discontinuity design comparing development density on opposite sides of ward boundaries finds that new construction is 10-16\% less dense on the stricter side of ward boundaries, with effects concentrated within 250 feet of the boundary. Second, event studies exploiting the 2012 announcement and 2015 implementation of ward redistricting show that home prices rise approximately 4\% when census blocks are reassigned to stricter aldermen, with effects beginning at announcement—suggesting that housing markets are forward-looking with respect to regulatory changes. Rents rise 3-5\% following reassignment to stricter aldermen, though these estimates are less precise.

These findings have several implications. First, within-city fragmentation of land use authority can generate coordination failures analogous to those documented across municipalities: individual decision-makers rationally restrict development, but the collective outcome is undersupply of housing relative to the social optimum. Second, the effects of regulatory stringency are highly localized—concentrated near ward boundaries where the jurisdiction of strict and lenient aldermen meet—suggesting that developers are attentive to aldermanic preferences at fine spatial scales. Third, housing markets capitalize expected future regulation, not just current policy, implying that credible reforms to aldermanic privilege could affect prices even before implementation.

Several limitations warrant mention. The estimates presented here do not capture the extensive margin ``deterrence'' effects and likely represent lower bounds of the true impact, as the considered projects that are never proposed are unobserved. The analysis focuses on local, partial-equilibrium effects and does not capture citywide general-equilibrium adjustments to changes in aldermanic strictness. And while the event study design supports causal interpretation, I cannot fully separate supply-side mechanisms (aldermen restricting construction) from demand-side mechanisms (stricter aldermen signaling neighborhood desirability) in explaining price effects.

Future work could extend this analysis in several directions: quantifying the welfare costs of aldermanic privilege using a structural model of developer and household location choice; examining whether aldermanic strictness affects the composition of housing supply, particularly affordable versus market-rate units; and studying whether electoral competition or transparency requirements discipline restrictive behavior. Extending the empirical approach to other cities with ward-based governance could assess the external validity of these findings.

Chicago's ongoing debate over aldermanic privilege is not merely symbolic. The institution has real, quantifiable effects on where housing gets built and how much it costs. As cities across the United States grapple with housing affordability crises, understanding how local political institutions shape housing supply is essential. The evidence presented here suggests that reforming fragmented land use authority through some combination of centralization, by-right development, or other mechanisms could meaningfully improve housing market outcomes.
