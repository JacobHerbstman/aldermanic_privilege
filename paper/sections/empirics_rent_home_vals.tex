\section{Event Study Evidence: Home Prices and Rents}
\label{sec:empirical_results_rent_home_vals}

\subsection{Event Study Design}
The cross-sectional evidence in Section \ref{sec:empirical_results_density} establishes that new construction is less dense on the stricter side of ward boundaries.
While density on its own is an important outcome, policymakers tend to be more interested in its implications for housing affordability.
To look at downstream effects on home values and rents, I exploit the 2015 and 2023 ward redistrictings as natural experiments to examine how home prices and rents respond when parcels are reassigned to aldermen with different levels of strictness.

\subsection{Identification Strategy and Empirical Specification}
\label{sec:identification_strategy}

Ward boundaries in Chicago are redrawn following each decennial census to equalize populations across the city's 50 wards. The 2015 remap (based on the 2010 Census) and the 2023 remap (based on the 2020 Census) each reassigned many parcels to new aldermen while leaving nearby parcels under the same ward. This creates a difference-in-differences style comparison: census blocks that switched aldermen (treated) versus nearby blocks that remained under the same alderman (control). The identifying assumption is that, absent the change in aldermanic representation, treated and control blocks would have followed parallel trends in housing costs.

This design exploits the fact that ward boundaries shift for reasons unrelated to housing market conditions in specific census blocks. Redistricting in Chicago occurs every ten years following the decennial census, with boundaries redrawn to equalize population across wards. The resulting boundary changes create quasi-random variation in alderman assignment for blocks near the old and new boundaries.

\paragraph{Announcement versus Implementation Timing.} A key feature of Chicago's redistricting process is that new ward maps are announced well in advance of taking effect. The 2015 ward boundaries were finalized in 2012, three years before the May 2015 aldermanic elections when the new map became effective. Similarly, the 2023 boundaries were finalized in 2022. If housing markets are forward-looking, prices may begin responding when redistricting maps are announced rather than when they are implemented.

For home sales, where transaction data extend back to 1999, I use the 2012 ward redistricting announcement as the event date for the 2015 remap. Since home prices take into account forward looking information about future restrictions on housing supply, announcement timing is more appropriate for capturing home price responses to exogenous shifts in the regulatory environment. Defining the treatment event at implementation shows substantial pre-trends in home prices prior to redistricting, confirming that 2012 is the more relevant treatment date for home sales.

As a result, my main estimates present results for home sales using announcement timing, which provides stronger support for the identifying assumption and suggests that housing markets incorporate information about future aldermanic representation once redistricting maps become public. For rental listings, data availability beginning in 2014 precludes analysis around the 2012 announcement. 

I therefore use implementation timing (2015 and 2023) for the rental analysis, stacking both redistricting events to maximize statistical power.\footnote{Appendix~\ref{sec:appendix_sales_rental} shows that results for home sales are qualitatively similar using implementation timing, though with less clean pre-trends due to the delay between announcement and implementation.}

\paragraph{Control Group Construction.} A treated block that switches from Ward $A$ to Ward $B$ could potentially be compared to control blocks that remained in either ward. However, blocks that remained in Ward $B$ were never subject to Ward $A$'s alderman, making them a poor counterfactual for how the treated block would have evolved absent redistricting. I therefore restrict the control group to blocks that remained in the \emph{origin} ward---that is, blocks near the same ward boundary that did not switch and thus experienced the same pre-treatment aldermanic policies as the treated blocks. This ensures that treated and control blocks share a common policy environment in the pre-period. To avoid contamination from electoral turnover, I also exclude control blocks in wards where the alderman changed via election (rather than redistricting) during the sample window.

Figure~\ref{fig:identification_illustration} illustrates this design. Panel A shows a concrete example for Wards 13 and 23 during the 2015 redistricting. The left subpanel shows ward assignments before redistricting, the center subpanel shows assignments after, and the right subpanel shows the resulting treatment status based on the change in alderman strictness. Blocks that switched from Ward 13 to Ward 23 are compared only to control blocks that remained in Ward 13, not to blocks that remained in Ward 23. Panels B and C show the citywide distribution of treatment and control blocks for the 2015 and 2023 redistricting events, respectively.

\paragraph{Empirical Specification.} I estimate the following event study specification at the transaction level (for sales) and listing level (for rentals):
%
\begin{equation}
\label{eq:event_study}
y_{it} = \alpha_{eps} + \lambda_{et} + \sum_{\tau=-5, \tau \neq -1}^{5} \beta_\tau \left( \Delta\text{Strictness}_i \times \mathbf{1}[t = \tau] \right) + \varepsilon_{it}
\end{equation}
%
\noindent where $y_{it}$ is the log sale price or log rent for transaction (or listing) $i$ at event time $t$. $\Delta\text{Strictness}_i$ is the change in alderman strictness for the block containing transaction $i$, defined as the strictness score of the destination alderman minus the origin alderman. To ensure treatment intensity is predetermined, strictness scores are measured in the year prior to redistricting (2014 for the 2015 redistricting cohort; 2022 for the 2023 cohort).

The key fixed effect $\alpha_{eps}$ is a cohort-by-ward-pair-by-origin-side fixed effect. This ensures that comparisons are made between treated and control transactions that (i) are near the same ward boundary, (ii) are in the same redistricting cohort, and (iii) were in the same origin ward prior to redistricting. $\lambda_{et}$ are cohort-by-year fixed effects, allowing for flexible time trends within each redistricting episode. Standard errors are clustered at the census block level to account for serial correlation within geographic units.

The coefficients $\beta_\tau$ trace out the effect of aldermanic strictness on housing costs in event time, with $\tau = -1$ (the year before redistricting, or its announcement for home sales) as the omitted reference period. The identifying assumption implies $\beta_\tau = 0$ for $\tau < 0$: prior to redistricting, there should be no systematic relationship between future changes in alderman strictness and housing cost trends within the same ward-pair-side cell.

I restrict the sample to transactions within 1,000 feet of ward boundaries to focus on locations where treated and control observations are most comparable in terms of neighborhood characteristics and local demand conditions. In the continuous treatment specification, $\Delta\text{Strictness}$ enters as a continuous variable; I also present binary specifications that allow for separate effects by treatment direction (moved to stricter versus moved to more lenient).

\begin{figure}[htbp]
    \centering
    % Panel A: Ward pair example (full width)
    \begin{minipage}{\textwidth}
        \centering
        \includegraphics[width=\textwidth]{../tasks/event_study_sales_diagnostics/output/ward_pair_before_after_13-23.pdf}
        \caption*{Panel A: Example of Identification Strategy (Ward Pair 13--23, 2015 Redistricting)}
    \end{minipage}
    
    \vspace{0.5cm}
    
    % Panel B and C: City-wide maps side by side
    \begin{minipage}{0.48\textwidth}
        \centering
        \includegraphics[width=\textwidth]{../tasks/event_study_sales_diagnostics/output/treatment_control_map_2015.pdf}
        \caption*{Panel B: Treatment and Control Blocks (2015)}
    \end{minipage}
    \hfill
    \begin{minipage}{0.48\textwidth}
        \centering
        \includegraphics[width=\textwidth]{../tasks/event_study_sales_diagnostics/output/treatment_control_map_2023.pdf}
        \caption*{Panel C: Treatment and Control Blocks (2023)}
    \end{minipage}
    
    \caption{Event Study Identification Strategy}
    \figurenotes{Panel A illustrates treatment assignment for ward pair 13--23 during the 2015 redistricting: ward assignments before (left), after (center), and treatment status based on the change in alderman strictness (right). Blocks switching from Ward 13 to Ward 23 are compared to controls that remained in Ward 13 (the origin ward), not to blocks that remained in Ward 23. Panels B and C show the citywide distribution of treated and control blocks for the 2015 and 2023 redistrictings. Sample restricted to blocks within 1,000 feet of ward boundaries. Controls exclude wards with aldermanic turnover via election.}
    \label{fig:identification_illustration}
\end{figure}

\paragraph{Predictions.} If stricter aldermen constrain housing supply, standard urban economics predicts that both home prices and rents should rise when blocks are reassigned to stricter aldermen. The event study design allows me to trace out the dynamics of this response and verify that effects emerge only after redistricting (or its announcement), supporting a causal interpretation.


Table~\ref{tab:summary_stats_event_study} presents summary statistics for the event study samples. 
The home sales sample includes $37,000$ transactions from the 2012 announcement cohort, while the rental sample includes $3.8$ million listings pooling the 2015 and 2023 implementation cohorts.
Treated and control groups are broadly similar on observable property characteristics: mean building size, age, and bedroom counts are comparable across treatment groups for home sales, as are unit sizes and building type distributions for rentals. 
This balance supports the identifying assumption that redistricting-induced changes in alderman assignment are orthogonal to pre-existing neighborhood characteristics.

\input{../tasks/summary_stats_event_study/output/summary_stats_event_study.tex}

\subsection{Results: Home Sale Prices}
\label{sec:results_home_prices}

Figure \ref{fig:event_study_sales} presents the main event study results for home sale prices. Panel A shows the continuous treatment specification, where treatment intensity is measured by the strictness score of the post-remap alderman. The coefficients trace out the effect of moving to an alderman who is one standard deviation stricter, relative to prices in the year before the remap.

Following the redistricting announcement, home prices in areas assigned to stricter aldermen diverge from the reference period. By year 5 after the announcement (and 2 years after the implementation of the new maps), moving to a ward with an alderman who is one standard deviation more ``strict'' is associated with approximately 4\% higher home prices. This effect is persistent, suggesting a permanent level shift rather than a transitory response.

Panel B presents the binary specification, which relaxes the linearity assumption and allows for separate effects by treatment direction. Here, treatment is defined as switching to an alderman with a higher ``strictness score'' (``moved to stricter''), or the reverse (``moved to more lenient''), with the control group consisting of census blocks that did not switch aldermen.
This specification is inherently noisier since it does not use variation in treatment intensity, but the pattern is qualitatively similar. Blocks that switched to stricter aldermen experience a roughly 5\% increase in home prices by year 5, while blocks that switched to more lenient aldermen see no significant change on average.

These results are robust to including property-level hedonic controls for building characteristics, as reported in Appendix~\ref{sec:appendix_sales_rental}. Point estimates are somewhat attenuated with hedonic controls, consistent with part of the effect operating through changes in housing stock composition induced by aldermanic supply restrictions.

\begin{figure}[htbp]
    \centering
    \begin{minipage}{0.48\textwidth}
        \centering
        \includegraphics[width=\textwidth]{../tasks/run_event_study_sales_disaggregate/output/event_study_disaggregate_yearly_unstacked_2012_continuous_uniform_1000ft_no_hedonics.pdf}
        \caption*{Panel A: Continuous Treatment}
    \end{minipage}
    \hfill
    \begin{minipage}{0.48\textwidth}
        \centering
        \includegraphics[width=\textwidth]{../tasks/run_event_study_sales_disaggregate/output/event_study_disaggregate_yearly_unstacked_2012_binary_uniform_1000ft_no_hedonics.pdf}
        \caption*{Panel B: Binary Treatment}
    \end{minipage}
    \caption{Event Study: Effect of Alderman Strictness on Home Sale Prices}
    \figurenotes{Panel A: continuous treatment (effect of 1 SD increase in strictness). Panel B: binary treatment (separate effects for blocks moving to stricter vs.\ more lenient aldermen). 
    Cohort-by-ward-pair-by-side and year fixed effects. Year $-1$ omitted as reference. Sample: transactions within 1,000 feet of ward boundaries. 95\% confidence intervals based on standard errors clustered at the census block level.}
    \label{fig:event_study_sales}
\end{figure}


\paragraph{Interpretation.} The event study estimates complement the cross-sectional density results from Section \ref{sec:empirical_results_density}. The border discontinuity estimates showed that new construction is 10--16\% less dense on the stricter side of ward boundaries. If supply restrictions bind, standard urban economics models predict that reduced housing supply should capitalize into higher prices. The event study confirms this prediction: census blocks exogenously reassigned to stricter aldermen experience rising home prices relative to nearby blocks that were not reassigned.

The magnitude is economically meaningful. A 4\% increase in home prices corresponds to approximately \$9,000 for the median home in my sample, representing a substantial transfer from buyers to incumbent homeowners. Combined with the density results, these findings suggest that aldermanic privilege has quantitatively important effects on both housing supply and affordability and allows incumbent homeowners to extract rents through political channels.

\subsection{Results: Rents}
\label{sec:results_rents}

Figure \ref{fig:event_study_rents} presents the analogous event study results for rental listings. Panel A shows the continuous treatment specification. The pattern is qualitatively similar to the home price results: pre-period coefficients, while noisy, are not significantly different from zero in the years immediately preceding alderman reassignment and then gradually rise over the post period. 
5 years after being reassigned to an alderman one standard deviation stricter, rents increase roughly $5-8\%$ on average relative to census blocks in the same ward-border pair that did not switch aldermen.

The rental estimates are less precise than the home price estimates and I cannot formally reject the null hypothesis of no effect at the $5\%$ level, but I can at the $10\%$ level. Nevertheless, the pre-period coefficients are stable around $0$, and the post-period coefficients steadily increase. This pattern is consistent with the gradual restriction of housing supply slowly matriculating into higher rents over time.

Panel B shows the binary specification for rents. While the estimates are noisier, the pattern is again consistent with a gradual divergence between blocks reassigned to stricter versus more lenient aldermen. While I cannot reject the effects are equivalent between blocks that switched to stricter versus more lenient aldermen, the same $\approx 5\%$ increase in rents still opens up by year 5 in the post-period. 

The rental results are more suggestive than definitive, but are robust to including listing-level hedonic controls as shown in Appendix~\ref{sec:appendix_sales_rental} and using different bandwidths and weighting schemes.
Combined with the evidence on home prices, these results suggest that alderman that induce more restrictions on housing supply through their aldermanic privilege ultimately raise housing costs and rents. 

Table~\ref{tab:did_sales} in Appendix~\ref{sec:appendix_sales_rental} presents pooled difference-in-differences estimates that summarize the average post-period effect, and table \ref{tab:did_rental} presents analogous results for rents. Across all specifications, moving to a stricter alderman is associated with higher housing costs.
Both tables include specifications with and without property-level hedonic controls, confirming that the results are not driven solely by changes in housing stock composition, though part of the effect does appear to operate through this channel, consistent with the density results in Section~\ref{sec:empirical_results_density}.


\begin{figure}[htbp]
    \centering
    \begin{minipage}{0.48\textwidth}
        \centering
        \includegraphics[width=\textwidth]{../tasks/run_event_study_rental_disaggregate/output/event_study_disaggregate_yearly_stacked_continuous_uniform_1000ft.pdf}
        \caption*{Panel A: Continuous Treatment}
    \end{minipage}
    \hfill
    \begin{minipage}{0.48\textwidth}
        \centering
        \includegraphics[width=\textwidth]{../tasks/run_event_study_rental_disaggregate/output/event_study_disaggregate_yearly_stacked_binary_direction_uniform_1000ft.pdf}
        \caption*{Panel B: Binary Treatment}
    \end{minipage}
    \caption{Event Study: Effect of Alderman Strictness on Rents}
    \figurenotes{Panel A: continuous treatment (effect of 1 SD increase in strictness). Panel B: binary treatment (separate effects for blocks moving to stricter vs.\ more lenient aldermen). Stacked estimator with cohort-by-ward-pair and event-by-year fixed effects. Year $-1$ omitted as reference. Sample: blocks within 1,000 feet of ward boundaries. 95\% confidence intervals based on standard errors clustered at the census block level.}
    \label{fig:event_study_rents}
\end{figure}
