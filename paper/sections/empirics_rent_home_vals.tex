\section{Event Study Evidence: Home Prices and Rents}
\label{sec:empirical_results_rent_home_vals}

\subsection{Event Study Design}
The cross-sectional evidence in Section \ref{sec:empirical_results_density} establishes that new construction is less dense on the stricter side of ward boundaries.
While density on its own is an important outcome, policymakers tend to be more interested in its implications for housing affordability.
To look at downstream effects on home values and rents, I exploit the 2015 and 2023 ward redistrictings as natural experiments to examine how home prices and rents respond when parcels are reassigned to aldermen with different levels of strictness.

\subsection{Identification Strategy and Empirical Specification}
Ward boundaries in Chicago are redrawn following each decennial census to equalize populations across the city's 50 wards. The 2015 remap (based on the 2010 Census) and the 2023 remap (based on the 2020 Census) each reassigned many parcels to new aldermen while leaving nearby parcels under the same ward. This creates a difference-in-differences style comparison: census blocks that switched aldermen (treated) versus blocks near the same ward boundary that did not switch (control). The identifying assumption is that, absent the change in aldermanic representation, treated and control blocks would have followed parallel trends in housing costs.

I conduct the analysis at the census block level rather than the parcel level for a practical reason: individual parcels rarely transact more than once within my sample window, making it impossible to construct a balanced panel at the parcel level. By aggregating to census blocks, I can track average prices and rents over time within small, relatively homogeneous geographic units while maintaining sufficient observations in each period.

If stricter aldermen constrain housing supply, standard urban economics predicts that both home prices and rents should rise in affected areas. I test this prediction using two complementary data sources: home sale prices from the Cook County Assessor's Office and rental listings from RentHub.

I estimate a stacked, weighted event study specification, pooling across both redistricting episodes for additional statistical power. 
Stacking also addresses concerns about heterogeneous treatment effects in standard two-way fixed effects estimators \citep{sun_estimating_2021,de_chaisemartin_two-way_2020}. Specifically, I run:

\begin{equation}
y_{bpet} = \alpha_b + \lambda_{et} + \sum_{\tau=-5, \tau \neq -1}^{5} \beta_\tau \left( \text{Strictness}_b \times \mathbf{1}[t = \tau] \right) + \varepsilon_{bpet}
\end{equation}

\noindent where $y_{bpet}$ is the outcome (log mean sale price or log mean rent) for census block $b$ along border-pair $p$ in event cohort $e$ (2015 or 2023 remap) at time $t$. $\text{Strictness}_b$ is the strictness score of the post-remap alderman for block $b$, as defined in Section \ref{sec:empirical_results_density}. In the continuous treatment specification, this enters as a continuous variable; I also present binary specifications below. $\alpha_b$ are census block fixed effects, which absorb time-invariant neighborhood characteristics. $\lambda_{et}$ are event-by-year fixed effects, allowing for different time trends across the two redistricting episodes. 
Standard errors are clustered at the census block level to account for serial correlation.

The coefficients $\beta_\tau$ trace out the effect of aldermanic strictness on housing costs in event time, with $\tau = -1$ (the year before each remap took effect) as the omitted reference period. The identifying assumption implies $\beta_\tau = 0$ for $\tau < 0$: prior to each remap, there should be no systematic relationship between future alderman strictness and housing cost trends.

I restrict the sample to census blocks where the average parcel distance to a ward boundary is within 1000 feet to focus on locations where treated and control blocks are most comparable in terms of neighborhood characteristics and local demand conditions.