\section{Literature Review}
\label{sec:lit_review}

My analysis relies on the unique institutional setup of Chicago, often called ``Aldermanic Privilege'' or ``Prerogative''. While the City Council technically votes on zoning changes, there is a powerful informal norm where the Council almost universally defers to the Alderman of the ward where a parcel is located \cite{Khan2021}. This effectively gives each Alderman veto power over development in their jurisdiction.

This matters because it means Chicago isn't operating as one unified city optimizing housing supply. It operates more like 50 non-cooperative jurisdictions. Developers who want to build denser housing---anything denser than the restrictive baseline zoning---have to negotiate directly with the Alderman. These officials are maximizing a political objective function that heavily weights local incumbents, particularly homeowners who are more likely to vote and oppose density, over the diffuse interests of future residents or the city's tax base.

\subsection{Literature Review}
I'm pulling from three main strands of literature here. First, there is the political economy of zoning. We know from \cite{Fischel2001} that homeowners act as ``homevoters'' to restrict supply. \cite{Khan2021} has done great work specifically on Chicago, showing that wards with higher homeownership rates enact stricter zoning and allow less density. My project takes those reduced-form findings and tries to build a structural microfoundation for them.

Second, I'm looking at the quantitative spatial economics literature on endogenous regulation. \cite{Parkhomenko2023} has a great model where local landowners vote on regulation to maximize land values, trading off agglomeration against congestion. I'm adapting that logic to a fragmented setting. I'm also really influenced by \cite{Bordeu2025}, who shows that fragmented municipalities under-invest in roads near their borders because the benefits spill over to neighbors. I'm applying a similar ``spillover'' logic to housing: wards under-supply density because the agglomeration and fiscal benefits ``leak'' out to the rest of the city.

Finally, I'm looking at which regulations actually bind. \cite{KulkaSoodChiumenti2024} use boundary discontinuities in Boston to show that density restrictions (like units per acre) are the binding constraint on prices, much more so than height limits. This validates my focus on ``development capacity'' as the key choice variable for local officials.
