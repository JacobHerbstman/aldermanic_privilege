\section{Data and Institutional Background}
\label{sec:data}

I construct a novel dataset of all new construction in Chicago linked to every alderman-pair episode since 2006 using various publicly-accessible datasets from the Cook County Assessor's Office.
In order to measure the density and frequency of new construction,
I first use data on Single and Multi-Family Improvement Characteristics which allows me to observe every ``improvement'' for each parcel in Cook County. 
I merge this with the ``Parcel Universe'' dataset which gives me precise location coordinates. 
This allows me to limit my sample to just parcels within the City of Chicago, and to assign each parcel to its respective ward using shapefiles from the City of Chicago's data portal.\footnote{https://data.cityofchicago.org/}
I then collapse this data to the first date I observe the parcel and consider that to be the construction date of the property, and create a cross-section of properties constructed between 2006 and 2025. 
Since this data is limited to parcels that have at most 6 units of housing, I then use the Commerical Valuation dataset from Cook County to infer multifamily housing density 
for buildings with more than 6 units and merge this with my residential conversion dataset to get the universe of new construction in Chicago between 2006 and 2025.

Other data products I use are the census and American Community Survey (ACS) data to create my independent variable, homeownership rate, at the ward level, and 
to control for various ward-level demographic characteristics in my regressions. I use zoning code data from 2nd city zoning \footnote{https://secondcityzoning.org/zones/} 
to include zoning code fixed effects in my regressions as well to ensure that changes in density at political borders do not come from mechanical changes in zoning. 

In Table~\ref{tab:summary_stats}, I present summary statistics for my sample of new residential construction. 
I present statistics for my full sample in column (1), and then for multifamily buildings in column (2). 
In total, there are 13,236 new residential buildings constructed in Chicago between 2006 and 2025, 
and 2,469 of them were multifamily buildings with 2 or more units.

The total new housing stock has an average of $6.5$ dwelling units and $2.3$ stories, with an average Floor Area Ratio (FAR) of $1.22$ and Dwelling Units Per Acre of $28.47$. 
Multifamily buildings are denser on average, with $31.3$ dwelling units, $2.8$ stories, a FAR of $2.3$, and $74.5$ dwelling units per acre. 
They are similar distances to their closest ward boundary at a median distance of $930$ feet overall and $887$ feet for multifamily buildings.
In addition, both have positive average ``strictness scores'' as defined in Section~\ref{sec:empirical_results_density}. 
While this could be evidence that my strictness scores are not caputuring pro vs. anti-development sentiment, 
it is also possible that demand is substantially higher in wards with stricter alderman, which is why the community elects them in an attempt to preserve ward character. 
In order to distentangle the effect of aldermanic influence from demand-side factors and neighborhood trends, more sophisticated empirical strategies are needed, as I discuss in Sections ~\ref{sec:empirical_results_density} and ~\ref{sec:empirical_results_rent_home_vals}.

\input{../tasks/summary_stats_new_construction/output/summary_stats.tex}

\vspace{-1em}
\noindent\footnotesize\textit{Notes:} Sample includes all new residential construction in Chicago, 2006--2024. FAR = Floor Area Ratio (building sqft / lot sqft). DUPAC = Dwelling Units Per Acre. Multifamily defined as buildings with 2 or more units. Distance to boundary measured from parcel centroid to nearest ward border. Strictness score is the standardized aldermanic strictness index for the parcel's ward.
\normalsize
