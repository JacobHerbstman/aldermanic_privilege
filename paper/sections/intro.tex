\section{Introduction}
\label{sec:intro}

Housing Prices are skyrocketing in major US cities and across the globe. 
Given the scarcity of vacant lots in dense urban areas, the primary way to add more housing supply is to construct taller 
and denser buildings that allow more people to live on the same-sized parcel of land. 
Density not only makes housing more affordable, but cities have long-known positive agglomeration spillovers from having people and firms in close proximity, 
implying indirect benefits on top of the direct benefits of lower housing prices \citep{rossihansberg_housing_2010,ahlfeldt_economic_2019,baum-snow_local_2024}.

However, recent research in economics and other fields has highlighted that local 
land-use regulations, which raise the cost of building denser housing, make housing more expensive and drive people out of neighborhoods they have lived in for generations \citep{duranton_urban_2023}.

Due to the ubiquitous nature of these regulations, this research has explored how these regulations come to be and the political economy surrounding them. 
A large strand of literature, pioneered by \cite{fischel_homevoter_2009}, discussed how local homeowners have incentives to lobby for restrictive zoning policies to protect their property values.
Additional research building on this idea has explored how local politics and the fragmentation of decision-making across municipalities can lead to under-provision of housing and other public goods with positive externalities
 \citep{hilber_origins_2013, ortalo-magne_political_2014, mast_warding_2024,bordeu_commuting_2025}.

While the majority of this research has focused on fragmentation across municipalities \cite{kulka_how_2022, monarrez_dividing_2023},
there is very little research on fragmentation within municipalities themselves. 
This mechanism can be equally as important as fragmentation across municipalities, especially in large cities where local political representation is often broken down into small districts or wards.

In this paper, I study the effect of within-municipality political fragmentation on housing supply in the context of the City of Chicago. 
While many cities have fragmented decision-making, no major city in the United States gives as much power to its city council members (Aldermen) as Chicago. 
Importantly, Chicago lacks a city charter that clearly lays out the powers and restrictions of various elected offices of the city, so much of the power of the Aldermen is derived from tradition and informal norms.
This unique level of power is often called ``Aldermanic Privilege,'' and it allows Aldermen to exert significant control over land-use decisions in their wards \citep{thale_aldermanic_2005}.
This privilege is so strong, in fact, that aldermen have been referred to as ``little mayors" ruling their Wards as fiefdoms \cite{einhorn_property_1991}.

While this power has existed for many decades, it only recently has come under scrutiny as Chicago has faced a housing affordability crisis similar to many other major cities, with the 
Department of Housing and Urban Development alleging that the power has been used to ensure affordable housing is not built in certain affluent wards \cite{chase_chicago_2023}.
In response to these allegations, Mayor Lightfoot signed an executive order attempting to limit the power of aldermanic privilege as part of a larger ethics reform package \cite{mayors_press_office_mayor_2019},
but in practice not much has changed in the years since \cite{cherone_developer_2025}. 
As a result, the debate continues today as to whether this power should be curtailed, or whether aldermen are still the best suited to make land-use decisions in their wards.\footnote{See this link, where each city council candidate was asked specifically about their stance on ``aldermanic prerogative": https://news.wttw.com/elections/voters-guide/2023/races/city-council}

To my knowledge, only one other paper has studied the effect of aldermanic privilege on housing supply in the city. 
\cite{khan_decentralized_2021} uses FOIA-requested rezoning data to look at the intensive-margin of aldermanic privilege, examining a specific function of the office where aldermen can personally approve or deny requests for zoning modifications on a specific lot. 
He finds that within 250 feet of a ward boundary, rezonings are $\approx 10\%$ smaller in terms of floor-area ratio (FAR), 
suggesting that aldermen are less willing to approve large rezonings when the benefits spill over into another ward, since the congestion costs are still likely to be internalized by the alderman's constituents even though the benefits are diffuse.

I expand on the work of \cite{khan_decentralized_2021} by focusing on the extensive-margin effects of aldermanic privilege. By looking at the density of all new construction in Chicago between 2006 and 2025 and not just specific rezoned properties, I can get a more holistic view of the aggregate effects of aldermanic privilege on housing supply in the city.
In addition, I want to use my empirical results to investigate the fiscal impacts of this fragmentation, as it is possible that the aggregate effects of this under-provision of housing could have significant implications for city finances through lost property tax revenue and other channels.
Given the city's structural budget deficit and stagnant population growth, examining policy levers by which policymakers can increase housing supply and tax revenue is of utmost importance.

My empirical strategy proceeds in two stages. First, I construct an ``aldermanic strictness score'' for each alderman using publicly available building permits data. This score captures the bureaucratic frictions that each alderman imposes on the development process, measured by permit processing times for high-discretion permits after residualizing out ward characteristics and applying empirical Bayes shrinkage. The resulting scores reveal substantial variation across aldermen that is not simply a reflection of geographic location or neighborhood demographics.

Second, I use these strictness scores to examine how development density varies at ward boundaries. Using a spatial regression discontinuity design that compares parcels on opposite sides of ward boundaries within the same zoning classification, I find that new construction is approximately 50-60\% less dense on the stricter side of ward boundaries. These visual discontinuities are large and statistically significant, though they likely reflect both the causal effect of aldermanic behavior and developer self-selection into more lenient wards.

To isolate the causal effect of aldermanic strictness, I employ a border-pair fixed effects design that compares new construction within 250 feet of ward boundaries, controlling for zoning classification and construction-year by border-pair fixed effects. I find that a one standard deviation increase in aldermanic strictness is associated with a 10\% decrease in dwelling units per acre, a 13\% decrease in floor-area ratio, and a 16\% decrease in total units. These effects attenuate substantially at greater distances from ward boundaries and the results are robust to a battery of specification checks including bandwidth sensitivity, donut-hole exclusions, alternative polynomial orders, and placebo boundary tests, as documented in Section~\ref{subsec:robustness}.
