\section{Robustness Checks}
\label{sec:appendixa}

\begin{table}[H]
    \caption{Border-Pair FE Estimates (bw = 1000 ft)}
    \label{tab:fe_estimates_1000}

    \input{../tasks/border_pair_FE_regressions/output/fe_table_bw1000.tex}
    
    \footnotesize
    \textit{Notes:} Standard errors clustered at the ward-pair level.
\end{table}


\subsection{Covariate Balance Test}

A key identifying assumption in the border-pair fixed effects design is that neighborhood characteristics vary smoothly across ward boundaries, so any discontinuity in development outcomes can be attributed to differences in aldermanic behavior rather than underlying area characteristics.

To test this assumption, I assign each parcel to a census block group via spatial intersection with 2019 American Community Survey (ACS) geographies. Block groups are small geographic units (typically 600--3,000 residents) that provide demographic variation at a much finer scale than wards. I then merge block group-level demographics from the five-year ACS estimates corresponding to each parcel's construction year (using 2010--2014 estimates for pre-2015 construction and 2015--2019 estimates for later construction).

I regress each demographic variable on the aldermanic strictness score using the same fixed effects specification as my main regressions (zone code $\times$ ward-pair and year fixed effects), restricting to parcels within 250 feet of the border.

\begin{table}[H]
    \caption{Identification Check: Covariate Balance at Census Block Group Level (bw = 250 ft)}
    \label{tab:fe_balance_250}
    
    \textit{Panel A: Race, Income, and Education}
    \vspace{0.2cm}
    
    \input{../tasks/border_pair_FE_regressions_validation/output/fe_validation_table_bw250_panel_a.tex}
    
    \vspace{0.5cm}
    
    \textit{Panel B: Housing and Density}
    \vspace{0.2cm}
    
    \input{../tasks/border_pair_FE_regressions_validation/output/fe_validation_table_bw250_panel_b.tex}
    
    \footnotesize
    \textit{Notes:} Each column regresses a block group-level demographic variable on the strictness score. Standard errors clustered at the ward-pair level.
\end{table}

\clearpage

\subsection{FE Donut Robustness}

As an additional robustness check for the border-pair fixed effects results, I exclude parcels within 25 or 50 feet of the ward boundary. This addresses concerns that observations precisely at the border may be unusual in ways that drive the results.

\begin{table}[H]
    \caption{Border-Pair FE Estimates --- Donut Robustness (bw = 250 ft, donut = 25 ft)}
    \label{tab:fe_donut_25}

    \input{../tasks/border_pair_FE_regressions_donut/output/fe_table_bw250_donut25.tex}
    
    \footnotesize
    \textit{Notes:} Excludes parcels within 25 ft of ward boundary. Standard errors clustered at the ward-pair level.
\end{table}

\begin{table}[H]
    \caption{Border-Pair FE Estimates --- Donut Robustness (bw = 250 ft, donut = 50 ft)}
    \label{tab:fe_donut_50}

    \input{../tasks/border_pair_FE_regressions_donut/output/fe_table_bw250_donut50.tex}
    
    \footnotesize
    \textit{Notes:} Excludes parcels within 50 ft of ward boundary. Standard errors clustered at the ward-pair level.
\end{table}

\subsection{RD Robustness Figures}

\subsubsection{Donut Hole Specification}

One concern with spatial RD designs is that observations precisely at the boundary may be unusual in ways that drive the results---for example, if ward boundaries tend to follow major streets or other geographic features that independently affect development patterns. To address this, I implement a ``donut RD'' specification that excludes parcels within 25 or 50 feet of the ward boundary. Figure~\ref{fig:rd_donut} shows that the discontinuity persists even when dropping observations closest to the cutoff, suggesting the results are not driven by idiosyncratic features of parcels located precisely on ward boundaries.

\begin{figure}[H]
    \centering
    % Log FAR donut
    \begin{minipage}{0.48\textwidth}
        \centering
        \includegraphics[width=\textwidth]{../tasks/spatial_rd_same_zone_only_donut/output/rd_plot_log_density_far_bw500_triangular_donut25.pdf}
    \end{minipage}
    \hfill
    \begin{minipage}{0.48\textwidth}
        \centering
        \includegraphics[width=\textwidth]{../tasks/spatial_rd_same_zone_only_donut/output/rd_plot_log_density_far_bw500_triangular_donut50.pdf}
    \end{minipage}
    
    \vspace{0.5cm}
    
    % Log DUPAC donut
    \begin{minipage}{0.48\textwidth}
        \centering
        \includegraphics[width=\textwidth]{../tasks/spatial_rd_same_zone_only_donut/output/rd_plot_log_density_dupac_bw500_triangular_donut25.pdf}
    \end{minipage}
    \hfill
    \begin{minipage}{0.48\textwidth}
        \centering
        \includegraphics[width=\textwidth]{../tasks/spatial_rd_same_zone_only_donut/output/rd_plot_log_density_dupac_bw500_triangular_donut50.pdf}
    \end{minipage}
    \caption{Donut Hole Robustness: Log(FAR) (top) and Log(DUPAC) (bottom) excluding parcels within 25ft (left column) and 50ft (right column) of the boundary}
    \label{fig:rd_donut}
\end{figure}

\subsubsection{Functional Form}

The baseline RD specification uses a local linear regression. However, results could be sensitive to this functional form assumption, particularly if the underlying relationship between distance and density is nonlinear. Figure~\ref{fig:rd_polynomial} presents estimates using linear, quadratic, and cubic polynomial specifications. The discontinuity is evident under all three specifications. Notably, the higher-order polynomials yield larger point estimates, suggesting that if anything, the linear specification provides a conservative estimate of the treatment effect.

\begin{figure}[H]
    \centering
    % Row 1: Linear (p=1)
    \begin{minipage}{0.48\textwidth}
        \centering
        \includegraphics[width=\textwidth]{../tasks/spatial_rd_same_zone_only_function_robustness/output/rd_log_density_far_linear_p1_bw500_triangular.pdf}
    \end{minipage}
    \hfill
    \begin{minipage}{0.48\textwidth}
        \centering
        \includegraphics[width=\textwidth]{../tasks/spatial_rd_same_zone_only_function_robustness/output/rd_log_density_dupac_linear_p1_bw500_triangular.pdf}
    \end{minipage}
    
    \vspace{0.3cm}
    
    % Row 2: Quadratic (p=2)
    \begin{minipage}{0.48\textwidth}
        \centering
        \includegraphics[width=\textwidth]{../tasks/spatial_rd_same_zone_only_function_robustness/output/rd_log_density_far_quadratic_p2_bw500_triangular.pdf}
    \end{minipage}
    \hfill
    \begin{minipage}{0.48\textwidth}
        \centering
        \includegraphics[width=\textwidth]{../tasks/spatial_rd_same_zone_only_function_robustness/output/rd_log_density_dupac_quadratic_p2_bw500_triangular.pdf}
    \end{minipage}
    
    \vspace{0.3cm}
    
    % Row 3: Cubic (p=3)
    \begin{minipage}{0.48\textwidth}
        \centering
        \includegraphics[width=\textwidth]{../tasks/spatial_rd_same_zone_only_function_robustness/output/rd_log_density_far_cubic_p3_bw500_triangular.pdf}
    \end{minipage}
    \hfill
    \begin{minipage}{0.48\textwidth}
        \centering
        \includegraphics[width=\textwidth]{../tasks/spatial_rd_same_zone_only_function_robustness/output/rd_log_density_dupac_cubic_p3_bw500_triangular.pdf}
    \end{minipage}
    \caption{Functional Form Robustness: Log(FAR) (left) and Log(DUPAC) (right) with Linear (p=1, top row), Quadratic (p=2, middle row), and Cubic (p=3, bottom row) specifications}
    \label{fig:rd_polynomial}
\end{figure}

\subsubsection{Placebo Boundary Tests}

A key identifying assumption of the RD design is that the discontinuity occurs precisely at the ward boundary and not at other arbitrary distances. To test this, I conduct placebo tests by artificially shifting the ``cutoff'' to locations 500 feet inside each ward (both toward the stricter and more lenient sides). If the discontinuity is truly driven by the ward boundary rather than some other spatial pattern, we should observe no discontinuity at these placebo cutoffs. Figure~\ref{fig:rd_placebo} confirms this prediction: the placebo boundaries show no significant discontinuity, while the true boundary (at zero) exhibits the large drop documented in the main results.

\begin{figure}[H]
    \centering
    % Log FAR placebo
    \begin{minipage}{0.48\textwidth}
        \centering
        \includegraphics[width=\textwidth]{../tasks/spatial_rd_same_zone_only_placebo_boundaries/output/placebo_rd_log_density_far_shift_minus500_bw500_triangular.pdf}
    \end{minipage}
    \hfill
    \begin{minipage}{0.48\textwidth}
        \centering
        \includegraphics[width=\textwidth]{../tasks/spatial_rd_same_zone_only_placebo_boundaries/output/placebo_rd_log_density_far_shift_plus500_bw500_triangular.pdf}
    \end{minipage}
    
    \vspace{0.5cm}
    
    % Log DUPAC placebo
    \begin{minipage}{0.48\textwidth}
        \centering
        \includegraphics[width=\textwidth]{../tasks/spatial_rd_same_zone_only_placebo_boundaries/output/placebo_rd_log_density_dupac_shift_minus500_bw500_triangular.pdf}
    \end{minipage}
    \hfill
    \begin{minipage}{0.48\textwidth}
        \centering
        \includegraphics[width=\textwidth]{../tasks/spatial_rd_same_zone_only_placebo_boundaries/output/placebo_rd_log_density_dupac_shift_plus500_bw500_triangular.pdf}
    \end{minipage}
    \caption{Placebo Boundary Tests: Log(FAR) (top) and Log(DUPAC) (bottom) at artificial cutoffs -500ft (left column) and +500ft (right column) from true boundary}
    \label{fig:rd_placebo}
\end{figure}