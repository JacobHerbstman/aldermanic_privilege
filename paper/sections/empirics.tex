\section{Empirical Results}
\label{sec:empirical_results}

\subsection{Measuring Aldermanic Frictions}

In order to examine the effects of aldermanic privilege I need a way to measure how much each alderman uses their privilege to influence development in their ward. 
I use publicly available building permits data to construct this measure by measuring how long average processing times for \textit{high-discretion} permits only. 
High-discretion permits are those designated as  ``New Construction, Renovation, Demolition, Porch Construction\footnote{While these may seem minor, they have received extra scrutiny since the 2003 balcony collapse.}'' permits and permits that were reinstated after being previously revoked. 
I have excluded permits in the easy permit process and express permit program, along with signs and scaffolding permits. 
The goal is to target the specific permit types that tend to draw community opposition and are more likely to face delays due to community meetings and aldermanic discretion on behalf of their constituents. 

Once I have my sample of permits, I construct the ``aldermen strictness score" as follows. 
First, I attempt to residualize out ward-specific characteristics by regressing the raw permit processing times 
on ward-level fundamentals such as median incomes, demographic characteristics, distance to the CBD and various other amenities, and month fixed-effects.
I then estimate aldermen fixed effects on these residualized processing times. 
Finally, since these estimates are a noisy signal of the ``true'' bureaucratic frictions caused by each alderman, 
I use empirical Bayes methods \cite{kline_discrimination_2024} to shrink these estimates towards zero based on how noisy each one is. 
Formally, the ``two-stage'' regression I run is: 
\begin{equation} \label{eq:strictness_reg1}
    \text{Processing Time}_{i} = \alpha + \mathbf{X}_w \beta + \gamma_{m} + \epsilon_{i} 
\end{equation}
\begin{equation} \label{eq:strictness_reg2}
    \hat{\epsilon}_{i} = \mu + \delta_{a} + \eta_{i}
\end{equation}

Where $\text{Processing Time}_{i}$ is the processing time of permit $i$, $\mathbf{X}_w$ is a vector of ward-level characteristics for the ward $w$ that permit $i$ is located in, and $\gamma_{m}$ are month fixed effects.
$\gamma_{a}$ are the aldermen fixed effects that I use as my dependent variable in my regressions. 
The scores are standardized to have mean zero and standard deviation one, and the results of my regressions can be found in Figure~\ref{fig:aldermanic_strictness_scores}.
In addition, a map of the aldermanic strictness scores can be found in Figure~\ref{fig:alderman_strictness_map}. 
There are strict and lenient aldermen all over the city, confirming I am not just picking up geographic patterns or neighborhood characteristics, but am actually capturing differences in the bureaucratic frictions created by each alderman.
In the context of the model in Section~\ref{sec:model}, I interpret alderman who have high strictness scores as having a lower $\lambda$ (local capture parameter) leading to lower housing supply in their wards.

\begin{figure}[htbp]
    \centering
    \includegraphics[width=0.8\textwidth]{../tasks/create_alderman_strictness_scores/output/Month_FEs_final_strictness_index.pdf} 
    \caption{Aldermanic Strictness Scores}
    \label{fig:aldermanic_strictness_scores}
\end{figure}

\begin{figure}[htbp]
    \centering
    \includegraphics[width=0.8\textwidth]{../tasks/strictness_score_map/output/strictness_score_map_2025-01.pdf} 
    \caption{Aldermanic Strictness Scores}
    \label{fig:alderman_strictness_map}
\end{figure}

\subsection{Visual Evidence of Density Discontinuities at Ward Borders}

Before moving to my formal causal results using border-pair fixed effects, I first present some visual evidence of discontinuities in development density at ward borders. 
I employ a regression discontinuity design with a linear control function and triangular weights to estimate the gap in development density of new construction at the ward cutoff. 
I stack all ward borders together to have enough power to get precise estimates by placing wards with more lenient aldermen on the left side of the border (negative distance) and wards with stricter aldermen on the right side of the border (positive distance).
I use the rdrobust package \cite{calonico_rdrobust_2015} to implement this design with bias-corrected point estimates and robust standard errors. 

In Figure~\ref{fig:rd_plots} I present regression discontinuity plots for my two primary measures of density, DUPAC and FAR.
I choose a bandwidth of 500 feet on either side of the border to focus on parcels that are the most likely to be similar in terms of neighborhood characteristics and local demand shocks, 
while still having enough observations to get precise estimates.
In order to make them comparable, I plot the log of each density measure and estimate the discontinuity at the border. 
I find large and statistically significant drops in both measures of density when moving from the lenient side of the border to the strict side.
Specifically, I estimate a $57\%$ decrease in DUPAC and a $51\%$ decrease in FAR when moving from the lenient side of the border to the strict side of a ward border.

They key assumption, as in any RD design, is that nothing else changes at the border except for the alderman that the developer has to work with.
If I were to claim causality from this design alone, I would need to claim that parcels cannot manipulate which side of the ward border they are on, 
but clearly developers can choose which ward to build in when considering development projects. 
This explains the large magnitude of the effects, as developers likely self-select into building in the more lenient wards if all other characteristics are equal between the two sides of the border.
Regardless, these results are still suggestive that aldermanic behavior is affecting development density in their wards and potentially driving down aggregate densities in the city. 
 
\begin{figure}[htbp]
    \centering
    \begin{minipage}{0.48\textwidth}
        \centering
        \includegraphics[width=\textwidth]{../tasks/spatial_rd_same_zone_only/output/rd_plot_log_density_dupac_bw500_triangular.pdf}
        \label{fig:rd_density_log_dupac}
    \end{minipage}
    \hfill
    \begin{minipage}{0.48\textwidth}
        \centering
        \includegraphics[width=\textwidth]{../tasks/spatial_rd_same_zone_only/output/rd_plot_log_density_far_bw500_triangular.pdf}
        \label{fig:rd_density_log_far}
    \end{minipage}
    \caption{RD Plots}
    \label{fig:rd_plots}
\end{figure}


\subsection{Do Stricter Aldermen Reduce Development Density?}

Armed with these ``strictness scores'' and the visual motivating evidence from Figure~\ref{fig:rd_plots}, I turn now to my main empirical question: do stricter aldermen reduce development density in their wards?

To answer this question I run regressions of the following form:

\begin{equation} \label{eq:boundary_reg}
    \ln(Y_{iwbt}) = \beta \cdot \text{Strictness}_{w} + \gamma \cdot |D_{i}| + \mathbf{X}_{w}'\Theta + \alpha_{z} + \mu_{bt} + \varepsilon_{iwbt}
\end{equation}

\noindent where $Y_{iwbt}$ represents the density outcome for parcel $i$ located in ward $w$ along border-pair $b$, constructed in year $t$. 

My three density outcomes of interest are common in the literature: Dwelling Units Per Acre (DUPAC), which is defined as the number of dwelling units on the parcel divided by the parcel's acreage; Floor Area Ratio (FAR), defined as the total building square footage on the parcel divided by the parcel's land area; and Units in the building.

The variable of interest is $\text{Strictness}_{w}$, which denotes the standardized bureaucratic friction score for the alderman of ward $w$. To control for spatial trends, we include $|D_{i}|$, the absolute distance of the parcel to the ward boundary. The vector $\mathbf{X}_{w}$ includes time-varying ward-level controls, including median household income, racial composition, and education levels.

Critically, I include a stringent set of fixed effects to isolate the causal effect of aldermanic strictness on development density.
$\alpha_z$ are zoning classification fixed effects, which ensure that comparisons are only made within the same zoning classification on each side of the border. 
For example, if one side of the border is zoned for RT-3.5, which allows a maximum FAR of 1.05, and the other side is zoned for RM-5.5, 
which allows for a maximum FAR of 2.5, there will be a mechanical difference in densities at the border that is not the result of aldermanic influence.

In addition, following \cite{black_better_1999,bayer_unified_2007,kulka_how_2022}, I include border-pair by construction-year fixed effects $\mu_{bt}$, which require that comparisons are only made between parcels that are located along the same border-pair and were constructed in the same year.
While this is very strict and limits the amount of variation in my sample, differential trends in development density across different parts of the city (i.e. the west loop boom) could otherwise confound my estimates. 

And finally, I limit the sample to parcels with between 2 and 100 units to focus on mid-size multifamily housing. 
Single family housing is the vast majority of new construction and does not have significant differences across political borders, and large developments (100+ units) are relatively rare and typically go through a more complicated planned development (PD) process with more oversight from the city and alderman.

The main results are presented in Table~\ref{tab:fe_estimates_250}. 
I limit the sample to parcels within 250 feet of a ward border to specifically test if different aldermen effect density at a very fine scale where local demand shocks and neighborhood characteristics are likely to be very similar on either side of the border.
The results are quite substantial and consistent across all three density measures. 
A one standard deviation increase in my aldermen strictness measure is associated with a 19\% decrease in DUPAC, a 12\% decrease in FAR, and a 13\% decrease in the number of units in the building, among new construction built in the same year and zoned the same on either side of the border.

In addition, to show that these differences are not driven by area characteristics and appear to be hyper-localized to the border where alderman differences are likely to have the most effect, 
I also run the same regressions but limit the sample to parcels within 1000 feet of a ward border. There is still some effect of aldermen strictness on DUPAC, but the magnitude is substantially smaller and I can reject that it is the same as in the 250 ft. bandwidth specification. 
The other coefficents are still negative but much smaller and not significantly different from zero, as shown in Table~\ref{tab:fe_estimates_1000}.
As a result, I interpret these results as strong evidence that aldermanic privilege and the bureaucratic frictions they create have a significant effect on development density in their wards.

I also remove the demographic controls from my regressions and run the same specifications, but with demographics as the outcome variables, in Table~\ref{tab:fe_balance_250} to see if aldermen strictness is correlated with changes in ward demographics that could be driving my results.
I find null results across the share of residents that are white, avarage household income, and the share of residents with a bachelor's degree or higher, suggesting that the differences in density at the ward borders are partly being driven by aldermenn behavior and not purely by demographic sorting.

\begin{table}[htbp]
    \caption{Border-Pair FE Estimates (bw = 250 ft)}
    \label{tab:fe_estimates_250}
    
    \input{../tasks/border_pair_FE_regressions/output/fe_table_bw250.tex}
    
    \footnotesize
    \textit{Notes:} Standard errors clustered at the ward-pair level.
\end{table}

\begin{table}[htbp]
    \caption{Border-Pair FE Estimates (bw = 250 ft)}
    \label{tab:fe_estimates_250}
    
    \input{../tasks/border_pair_FE_rental_regressions/output/fe_table_bw250.tex}
    
    \footnotesize
    \textit{Notes:} Standard errors clustered at the ward-pair level.
\end{table}