\section{Discussion}
\label{sec:discussion}

\subsection{Interpretation of Estimates}

It is useful to distinguish three margins along which aldermanic strictness can affect housing supply. First, the \textit{intensive margin}: conditional on a developer seeking discretionary approval (such as a rezoning or planned development designation), how does aldermanic strictness affect the size and density of the approved project? This is the margin studied by \cite{khan_decentralized_2021}, who finds that rezonings are approximately 10\% smaller near ward boundaries where benefits spill over to adjacent wards.

Second, the \textit{extensive margin of realized development}: how does aldermanic strictness affect what actually gets built, including projects that proceed by-right without seeking discretionary approval? This is the margin I study. By examining all new construction—not just rezoned parcels—I capture an additional channel: developers who respond to strict aldermen by building smaller by-right projects that do not require aldermanic sign-off, rather than proposing larger projects that would trigger discretionary review. My density estimates thus reflect both the intensive-margin effect documented by Khan (stricter aldermen approve smaller projects) and an extensive-margin effect (developers in strict wards avoid discretionary review entirely by building smaller).

Third, the \textit{extensive margin of deterred development}: projects that are never proposed because developers anticipate rejection, face prohibitive uncertainty, or choose to invest in other jurisdictions entirely. This margin is by definition unobserved in permit or construction data. If developers avoid strict wards at the site-selection stage—before any permit application is filed—then even my estimates of realized construction understate the true effect of aldermanic strictness on housing supply.

The estimates presented in this paper capture the first two margins but not the third. They are therefore best interpreted as \textit{lower bounds} of the total effect of aldermanic privilege on housing supply. Several mechanisms contribute to the unmeasured extensive margin of deterred development.

First, regulatory uncertainty deters risk-averse developers even when expected approval probability is reasonable. A developer considering a 50-unit apartment building may abandon the project if there is substantial probability of rejection, even if the most likely outcome is approval. The option value of waiting or building elsewhere may dominate proceeding with an uncertain approval process. 

Second, the negotiation process itself imposes costs. Aldermen may approve projects conditional on design modifications, density reductions, or community benefit agreements that are not reflected in final permit data. The anticipation of these costs may deter marginal projects from being proposed in the first place, even if they would ultimately be approved in some modified form.

Third, the mere existence of aldermanic discretion creates a ``chilling effect'' on development proposals. Developers report that navigating aldermanic approval is among the most uncertain and time-consuming aspects of Chicago development, leading some to focus their investments in suburban jurisdictions with more predictable approval processes. This reallocation of development activity away from the city represents a real cost that does not appear in any permit database.

These unmeasured channels suggest that the welfare costs of aldermanic privilege extend beyond what my reduced-form estimates can capture. A complete accounting would require modeling developer entry and site selection decisions, which I leave to future work.

\subsection{Connection to the Broader Literature}

My findings contribute to several strands of the urban economics literature. Most directly, this paper adds to the extensive evidence that land use regulations restrict housing supply and raise prices \citep{glaeser_why_2005, saiz_geographic_2010, turner_land_2014}. The contribution here is to show that this mechanism operates at the \textit{hyper-local} level—within a single city, across ward boundaries—not only across metropolitan areas or municipalities. Ward-level variation in regulatory stringency creates meaningful supply distortions even within a unified municipal boundary.

The results also speak to the political economy of land use regulation. \cite{fischel_homevoter_2009} argues that homeowners have strong incentives to support restrictive zoning to protect their property values. My price results are consistent with this mechanism: home prices rise approximately 5\% when blocks are reassigned to stricter aldermen. But this price appreciation represents a \textit{transfer} from prospective buyers to incumbent homeowners, not a social welfare gain. Aldermanic privilege thus functions as a mechanism through which homeowners can extract rents via political channels, with costs borne by prospective residents, renters, and the city as a whole through reduced tax base and diminished agglomeration economies.

More broadly, this paper contributes to a growing literature on how the fragmentation of governance affects urban outcomes. Prior work has emphasized fragmentation \textit{across} municipalities: Tiebout sorting, fiscal zoning, and the tendency for suburban jurisdictions to externalize the costs of housing restrictions onto regional housing markets \citep{hilber_origins_2013, kulka_how_2022, monarrez_dividing_2023, bordeu_commuting_2025}. My results demonstrate that fragmentation \textit{within} cities can generate similar coordination failures. Chicago is nominally a unified municipality with centralized zoning authority, but the informal institution of aldermanic privilege effectively Balkanizes land use decision-making across 50 independent actors.

This fragmentation creates a classic collective action problem. Housing affordability is a citywide public good: the benefits of new construction—lower prices, expanded tax base, agglomeration spillovers—are diffuse and accrue across the metropolitan area. But the costs of new development—construction disruption, traffic congestion, neighborhood change—are concentrated and borne primarily by residents of the ward where construction occurs. Each alderman, rationally maximizing the welfare of their constituents, will restrict development to a greater degree than a social planner who internalizes citywide benefits. The result is systematic undersupply of housing relative to the social optimum, even though no individual alderman is behaving irrationally.

This interpretation is consistent with \cite{khan_decentralized_2021}, who finds that aldermen approve smaller rezonings near ward boundaries where benefits spill over to adjacent wards. My paper complements his work by documenting the aggregate consequences of this behavior along a broader margin: not just smaller individual projects conditional on seeking approval, but systematically lower realized density and higher prices in wards governed by stricter aldermen.

\subsection{Policy Implications}

The results of this paper inform ongoing debates about reforming aldermanic privilege in Chicago. Mayor Lightfoot's 2019 executive order attempted to limit aldermanic discretion over development approvals, and the U.S. Department of Housing and Urban Development has raised concerns that the institution has been used to block affordable housing in affluent wards \citep{chase_chicago_2023}. My findings suggest these concerns have empirical foundation: aldermanic strictness has quantifiable effects on both housing supply and prices.

Several policy approaches could address the coordination failures documented here. One option is to \textit{centralize} land use authority, shifting discretion from individual aldermen to citywide bodies such as the Plan Commission or Zoning Board of Appeals. Centralized decision-makers would internalize the citywide benefits of housing production that individual aldermen rationally ignore. The tradeoff is that centralization sacrifices local knowledge—aldermen often have legitimate information about neighborhood conditions and constituent preferences that citywide bodies lack—and may reduce democratic accountability for land use decisions.

A second approach is to expand \textit{by-right} development, reducing the scope of aldermanic discretion for projects that comply with existing zoning. If a proposed building meets all code requirements, approval would be automatic rather than subject to aldermanic review. This approach preserves the zoning code as the mechanism for expressing community preferences about land use while eliminating the uncertainty and delay associated with discretionary review. The limitation is that by-right development reduces flexibility to address legitimate site-specific concerns that may not be anticipated by general zoning categories.

A third approach focuses on \textit{transparency}, requiring aldermen to publicly state their positions on pending development applications and creating systematic records of approval and rejection decisions. Greater transparency could enable electoral accountability: voters who prioritize housing affordability could identify and oppose aldermen with restrictive track records. However, transparency alone may not change underlying political incentives if ward electorates are dominated by homeowners who benefit from restriction.

These approaches are not mutually exclusive, and the optimal policy likely involves some combination. The key insight from this paper is that the \textit{status quo} is costly. Chicago's current institution of aldermanic privilege generates measurable distortions to housing supply and prices. Whether the benefits of local control—democratic accountability, local knowledge, responsiveness to neighborhood concerns—outweigh these costs is ultimately a political judgment. But that judgment should be informed by evidence on the magnitude of the tradeoffs involved.