\begin{frame}
    \frametitle{Data Sources}
	\begin{itemize}
        % Item 1: Appears on slide 1 and stays visible.
        \item<1-> \textbf{Attom Data}
        % Details for item 1: Only appear on slide 1.
        \only<1>{
            \begin{itemize} 
                \item I use parcel-level data from Attom to create a sample of all new residential developments in Chicago from 2006-2023
                \item Contains exact geocoded location, lot size, square footage, number of units, and many other characteristics
                \item My final analysis sample contains $\approx 15,000$ unique developments
            \end{itemize}
        }
        \vspace{1em} 

        % Item 2: Appears on slide 2 and stays visible.
        \item<2-> \textbf{Chicago Building Permits Data}
        % Details for item 2: Only appear on slide 2.
        \only<2>{
            \begin{itemize}
                \item To create alderman fixed effects for the RD design, I use publicly-available permit data from the City of Chicago
                \item Includes over $800,000$ permits from 2006-2023, with exact geocoded location and permit type
                \item I also obtain ward shapefiles from the City of Chicago Data Portal
            \end{itemize}
        }
        \vspace{1em}
    
        % Item 3: Appears on slide 3 and stays visible.
        \item<3-> \textbf{Demographic Data}
        % Details for item 3: Only appear on slide 3.
        \only<3>{
            \begin{itemize}
                \item I obtain demographic variables at the block-group level from the American Community Survey (ACS)
                \item Used to create ward-level control variables for alderman fixed-effects regressions
                \item Include median income, population density, homeownership rate, and racial composition of each ward
            \end{itemize}
        }
        % No vspace needed after the last item, but you can add it if you want.
    \end{itemize}
\end{frame}

\begin{frame}
    \frametitle{ATTOM Parcel Data} 
        \begin{figure}
            \centering
            \includegraphics[width=\textwidth,height=0.82\textheight,keepaspectratio]{../tasks/new_construction_maps_plots/output/construction_map_2006_2014_w_density.pdf}
            \label{fig:alderman_effects}
        \end{figure}
 \end{frame} 






